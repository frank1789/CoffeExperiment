\chapter{Conclusione}
Entrambi gli esperimenti mostrano dei punti di contatto infatti dai grafici delle iterazioni è possibile osservare che alcune di esse sono in comune, quindi è possibile osservare che per la preparazione di un buon caffè è necessario utilizzare una quantità d'acqua vicino “alla valvola” e da preferire un quantità di caffe uniforme. Inoltre l'operazione di pressatura rende scarsa la resa della bevanda in termini di quantità, specialmente quando la quantità di polvere è elevata, quindi è preferibile evitarla.
 Si evince che la forza della fiamma ha effeti sulla tipo di acqua e questo si riperquote sulla capità di passare attraverso il caffè migliorando la resa, al contrario non mostra un effetto significativo sulla durata dell'intero processo.
